% Template for PLoS
% Version 3.1 February 2015
%
% To compile to pdf, run:
% latex plos.template
% bibtex plos.template
% latex plos.template
% latex plos.template
% dvipdf plos.template
%
% % % % % % % % % % % % % % % % % % % % % %
%
% -- IMPORTANT NOTE
%
% This template contains comments intended 
% to minimize problems and delays during our production 
% process. Please follow the template instructions
% whenever possible.
%
% % % % % % % % % % % % % % % % % % % % % % % 
%
% Once your paper is accepted for publication, 
% PLEASE REMOVE ALL TRACKED CHANGES in this file and leave only
% the final text of your manuscript.
%
% There are no restrictions on package use within the LaTeX files except that 
% no packages listed in the template may be deleted.
%
% Please do not include colors or graphics in the text.
%
% Please do not create a heading level below \subsection. For 3rd level headings, use \paragraph{}.
%
% % % % % % % % % % % % % % % % % % % % % % %
%
% -- FIGURES AND TABLES
%
% Please include tables/figure captions directly after the paragraph where they are first cited in the text.
%
% DO NOT INCLUDE GRAPHICS IN YOUR MANUSCRIPT
% - Figures should be uploaded separately from your manuscript file. 
% - Figures generated using LaTeX should be extracted and removed from the PDF before submission. 
% - Figures containing multiple panels/subfigures must be combined into one image file before submission.
% For figure citations, please use "Fig." instead of "Figure".
% See http://www.plosone.org/static/figureGuidelines for PLOS figure guidelines.
%
% Tables should be cell-based and may not contain:
% - tabs/spacing/line breaks within cells to alter layout or alignment
% - vertically-merged cells (no tabular environments within tabular environments, do not use \multirow)
% - colors, shading, or graphic objects
% See http://www.plosone.org/static/figureGuidelines#tables for table guidelines.
%
% For tables that exceed the width of the text column, use the adjustwidth environment as illustrated in the example table in text below.
%
% % % % % % % % % % % % % % % % % % % % % % % %
%
% -- EQUATIONS, MATH SYMBOLS, SUBSCRIPTS, AND SUPERSCRIPTS
%
% IMPORTANT
% Below are a few tips to help format your equations and other special characters according to our specifications. For more tips to help reduce the possibility of formatting errors during conversion, please see our LaTeX guidelines at http://www.plosone.org/static/latexGuidelines
%
% Please be sure to include all portions of an equation in the math environment.
%
% Do not include text that is not math in the math environment. For example, CO2 will be CO\textsubscript{2}.
%
% Please add line breaks to long display equations when possible in order to fit size of the column. 
%
% For inline equations, please do not include punctuation (commas, etc) within the math environment unless this is part of the equation.
%
% % % % % % % % % % % % % % % % % % % % % % % % 
%
% Please contact latex@plos.org with any questions.
%
% % % % % % % % % % % % % % % % % % % % % % % %

\documentclass[10pt,letterpaper]{article}
\usepackage[top=0.85in,left=2.75in,footskip=0.75in]{geometry}

% Use adjustwidth environment to exceed column width (see example table in text)
\usepackage{changepage}

% Use Unicode characters when possible
\usepackage[utf8]{inputenc}

% textcomp package and marvosym package for additional characters
\usepackage{textcomp,marvosym}

% fixltx2e package for \textsubscript
\usepackage{fixltx2e}

% amsmath and amssymb packages, useful for mathematical formulas and symbols
\usepackage{amsmath,amssymb}

% cite package, to clean up citations in the main text. Do not remove.
\usepackage{cite}

% Use nameref to cite supporting information files (see Supporting Information section for more info)
\usepackage{nameref,hyperref}

% line numbers
\usepackage[right]{lineno}

% ligatures disabled
\usepackage{microtype}
\DisableLigatures[f]{encoding = *, family = * }

% rotating package for sideways tables
\usepackage{rotating}

% Remove comment for double spacing
%\usepackage{setspace} 
%\doublespacing

% Text layout
\raggedright
\setlength{\parindent}{0.5cm}
\textwidth 5.25in 
\textheight 8.75in

% Bold the 'Figure #' in the caption and separate it from the title/caption with a period
% Captions will be left justified
\usepackage[aboveskip=1pt,labelfont=bf,labelsep=period,justification=raggedright,singlelinecheck=off]{caption}

% Use the PLoS provided BiBTeX style
\bibliographystyle{plos2015}

% Remove brackets from numbering in List of References
\makeatletter
\renewcommand{\@biblabel}[1]{\quad#1.}
\makeatother

% Leave date blank
\date{}

% Header and Footer with logo
\usepackage{lastpage,fancyhdr,graphicx}
\usepackage{epstopdf}
\pagestyle{myheadings}
\pagestyle{fancy}
\fancyhf{}
\lhead{\includegraphics[width=2.0in]{PLOS-submission.eps}}
\rfoot{\thepage/\pageref{LastPage}}
\renewcommand{\footrule}{\hrule height 2pt \vspace{2mm}}
\fancyheadoffset[L]{2.25in}
\fancyfootoffset[L]{2.25in}
\lfoot{\sf PLOS}

%% Include all macros below

\newcommand{\lorem}{{\bf LOREM}}
\newcommand{\ipsum}{{\bf IPSUM}}

%% END MACROS SECTION


\begin{document}
\vspace*{0.35in}

% Title must be 250 characters or less.
% Please capitalize all terms in the title except conjunctions, prepositions, and articles.
\begin{flushleft}
{\Large
\textbf\newline{iDynoMiCS 2: Fully flexible computer modelling platform for microbial ecology}
}
\newline
% Insert author names, affiliations and corresponding author email (do not include titles, positions, or degrees).
\\
Robert J Clegg\textsuperscript{1,2,3,*},
Bastiaan J R Cockx\textsuperscript{4},
Stefan Lang\textsuperscript{5},
Barth F Smets\textsuperscript{4},
Jan-Ulrich Kreft\textsuperscript{1,2,3},
\\
\bigskip
\bf{1} Centre for Computational Biology, University of Birmingham, Edgbaston, Birmingham B12 5TT, United Kingdom
\\
\bf{2} Institute of Microbiology and Infection, University of Birmingham, Edgbaston, Birmingham B12 5TT, United Kingdom
\\
\bf{3} School of Biosciences, University of Birmingham, Edgbaston, Birmingham B12 5TT, United Kingdom
\\
\bf{4} Department of Environmental Engineering, Technical University of Denmark, Bygningstorvet 115, 2800 Kgs. Lyngby, Denmark 
\\
\bf{5} Department of Bioinformatics, Friedrich Schiller University, Ernst-Abbe-Platz 2, 07743 Jena, Germany 
\\
\bigskip

% Insert additional author notes using the symbols described below. Insert symbol callouts after author names as necessary.
% 
% Remove or comment out the author notes below if they aren't used.
%
% Primary Equal Contribution Note
%\Yinyang These authors contributed equally to this work.

% Additional Equal Contribution Note
% Also use this double-dagger symbol for special authorship notes, such as senior authorship.
%\ddag These authors also contributed equally to this work.

% Current address notes
%\textcurrency a Insert current address of first author with an address update
% \textcurrency b Insert current address of second author with an address update
% \textcurrency c Insert current address of third author with an address update

% Deceased author note
%\dag Deceased

% Group/Consortium Author Note
%\textpilcrow Membership list can be found in the Acknowledgments section.

% Use the asterisk to denote corresponding authorship and provide email address in note below.
* r.j.clegg@bham.ac.uk

\end{flushleft}
% Please keep the abstract below 300 words
\section*{Abstract}


The source code is publicly available to download from \url{https://github.com/roughhawkbit/iDynoMiCS-2}.
% TODO transfer ownership of the repository to Jan's account, and make it public

% Please keep the Author Summary between 150 and 200 words
% Use first person. PLOS ONE authors please skip this step. 
% Author Summary not valid for PLOS ONE submissions.   
\section*{Author Summary}
% TODO

\linenumbers

\section*{Introduction}
Individual-based modelling of microbial communities is becoming ever more widespread as software is developed and computational power increases \cite{Ferrer2008}.
Drive for more generic software:
checked by more people, so more reliable \cite{Joppa2013};
easier to compare different model formulations, i.e. structural sensitivity  \cite{Adamson2013}.
Software based on \cite{Kreft1998, Lardon2011, Storck2014}. 

% Here it is really important to also highlight to advantages over other modeling frameworks out there, also in comparison with iDynoMiCS 1.0.
% Why do we need a new framework, what are it's selling points. We should be specific and go beyond the advantages of individual based models.

% written it in this form now, but maybe we should start by highlighting the benefits of iDynoMiCS 2 instead of starting with the limitations of iDynoMiCS 1
The original iDynoMiCS framework \cite{Lardon2011} has seen a rise in popularity and is now used by many research groups all over the world. Half a decade later, we have initiated a rigorous overhaul of the framework. This is done to address several evident limitations of the original package: Learning how to use iDynoMiCS can be challenging and multiple 3rd party tools are required to properly use the software, simulation protocols are sensitive to typographical errors and difficult to validate, it is not possible to simulate more than a single environment, it is difficult to combine agent types, only spherical agents can be modeled, agents can only interact with a single solute grid cell and it is not possible to evaluate processes at different timescales.

To address these issues the iDynoMiCS 2 framework has been developed. This overhauled version of iDynoMiCS allows the user construct a new and unique model by combining simple building blocks that govern the processes and behavior of the agents and their environment. The user can use the default building blocks or develop own building blocks which are easily to implement because of the modular structure and provided Java interfaces. A graphical user interface has been developed to assist the user with the model construction. This lowers the complexity for new iDynoMiCS users and reduces the changes of creating a protocol with errors.

% TODO incorporate selling points in introduction
%Selling points: 
%- easy to use
%- Highly detailed model setup
%- customization to specific use cases, adjustability 
%- fineness and accuracy
%- (correctness) % this one will need a lot of justification.
%- (computational speed) % also needs justification, this point will probably only hold with a good implementation of concurrent evaluation (on todo list)
% there are other IbMs out there that focus on speed, but to my current knowledge these models are far less extensive than iDynoMiCS 2.0

\section*{Models}
% You may title this section "Methods" or "Models". 
% "Models" is not a valid title for PLoS ONE authors. However, PLoS ONE
% authors may use "Analysis" 
The model description follows the ODD (Overview, Design concepts, Details) protocol \cite{Grimm2006, Grimm2010}. Further details are available in Supporting Information for some parts of the model description.

\subsection*{Purpose}
% Every model has to start from a clear question, problem, or hypothesis. Therefore, ODD starts with a concise summary of the overall objective(s) for which the model was developed. Do not describe anything about how the model works here, only what it is to be used for. We encourage authors to use this paragraph independently of any presentation of the purpose in the introduction of their article, since the ODD protocol should be complete and understandable by itself and not only in connection with the whole publication (as it is also the case for figures, tables and their legends). If one of the purposes of a model is to expand from basic principles to richer representation of real-world scenarios, this should be stated explicitly.

\subsection*{Entities, state variables, and scales}
% What kinds of entities are in the model? By what state variables, or attributes, are these entities characterized? What are the temporal and spatial resolutions and extents of the model? 
% See Grimm et al (2010) for further explanation

% TODO Compartments: shapes, environment & agent containers
% TODO Agent & aspect types
% TODO Process Managers

\subsection*{Process overview and scheduling}
% Who (i.e., what entity) does what, and in what order? When are state variables updated? How is time modeled, as discrete steps or as a continuum over which both continuous processes and discrete events can occur? Except for very simple schedules, one should use pseudo-code to describe the schedule in every detail, so that the model can be re-implemented from this code. Ideally, the pseudo-code corresponds fully to the actual code used in the program implementing the ABM.
% See Grimm et al (2010) for further explanation
Each compartment is self-contained.
No interaction between compartments within a global time-step, all transfers are stored and happen instantly between timesteps.

Process managers have own timestep; Compartment manages the order in which they are executed.

\subsection*{Design concepts}
The iDynoMiCS 2 structure inherits many concepts from previous packages \cite{Kreft1998, Lardon2011}, but also builds on these.

\paragraph{Basic principles}
% Which general concepts, theories, hypotheses, or modeling approaches are underlying the model’s design? Explain the relationship between these basic principles, the complexity ex-panded in this model, and the purpose of the study. How were they taken into account? Are they used at the level of submodels (e.g., decisions on land use, or foraging theory), or is their scope the system level (e.g., intermediate disturbance hypotheses)? Will the model provide insights about the basic principles themselves, i.e. their scope, their usefulness in real-world scenarios, validation, or modification (Grimm, 1999)? Does the model use new, or previously developed, theory for agent traits from which system dynamics emerge (e.g., ‘individual-based theory’ as described by Grimm and Railsback [2005; Grimm et al., 2005])?

\paragraph{Emergence}
% What key results or outputs of the model are modeled as emerging from the adaptive traits, or behaviors, of individuals? In other words, what model results are expected to vary in complex and perhaps unpredictable ways when particular characteristics of individuals or their environment change? Are there other results that are more tightly imposed by model rules and hence less dependent on what individuals do, and hence ‘built in’ rather than emergent results? 

\paragraph{Adaptation}
% What adaptive traits do the individuals have? What rules do they have for making decisions or changing behavior in response to changes in themselves or their environment? Do these traits explicitly seek to increase some measure of individual success regarding its objectives (e.g., “move to the cell providing fastest growth rate”, where growth is assumed to be an indicator of success; see the next concept)? Or do they instead simply cause individuals to reproduce observed behaviors (e.g., “go uphill 70% of the time”) that are implicitly assumed to indirectly convey success or fitness?
Agents may use species library for defaults, but overwrite these due to mutation, etc. Behaviours ("events") may depend on values of other local aspects.

\paragraph{Objectives}
% If adaptive traits explicitly act to increase some measure of the individual's success at meeting some objective, what exactly is that objective and how is it measured? When individuals make decisions by ranking alternatives, what criteria do they use? Some synonyms for ‘objectives’ are ‘fitness’ for organisms assumed to have adaptive traits evolved to provide reproductive success, ‘utility’ for economic reward in social models or simply ‘success criteria’. (Note that the objective of such agents as members of a team, social insects, organs—e.g., leaves—of an organism, or cells in a tissue, may not refer to themselves but to the team, colony or organism of which they are a part.) 

\paragraph{Learning}
% Many individuals or agents (but also organizations and institutions) change their adaptive traits over time as a consequence of their experience? If so, how?  
% TODO Leave out this bit?

\paragraph{Prediction}
% Prediction is fundamental to successful decision-making; if an agent’s adaptive traits or learning procedures are based on estimating future consequences of decisions, how do agents predict the future conditions (either environmental or internal) they will experience? If appropriate, what internal models are agents assumed to use to estimate future conditions or consequences of their decisions? What tacit or hidden predictions are implied in these internal model assumptions? 
% TODO Leave out this bit?

\paragraph{Sensing}
% What internal and environmental state variables are individuals assumed to sense and consider in their decisions? What state variables of which other individuals and entities can an individual perceive; for example, signals that another individual may intentionally or uninten-tionally send? Sensing is often assumed to be local, but can happen through networks or can even be assumed to be global (e.g., a forager on one site sensing the resource levels of all other sites it could move to). If agents sense each other through social networks, is the structure of the network imposed or emergent? Are the mechanisms by which agents obtain information modeled explicitly, or are individuals simply assumed to know these variables?

\paragraph{Interaction}
% What kinds of interactions among agents are assumed? Are there direct interactions in which individuals encounter and affect others, or are interactions indirect, e.g., via competition for a mediating resource? If the interactions involve communication, how are such communications represented?


\paragraph{Stochasticity}
% What processes are modeled by assuming they are random or partly random? Is stochasticity used, for example, to reproduce variability in processes for which it is unimportant to model the actual causes of the variability? Is it used to cause model events or behaviors to occur with a specified frequency?

\paragraph{Collectives}
% Do the individuals form or belong to aggregations that affect, and are affected by, the individuals? Such collectives can be an important intermediate level of organization in an ABM; examples include social groups, fish schools and bird flocks, and human networks and organizations. How are collectives represented? Is a particular collective an emergent property of the individuals, such as a flock of birds that assembles as a result of individual behaviors, or is the collective simply a definition by the modeler, such as the set of individuals with certain properties, defined as a separate kind of entity with its own state variables and traits?

\paragraph{Observation}
% What data are collected from the ABM for testing, understanding, and analyzing it, and how and when are they collected? Are all output data freely used, or are only certain data sampled and used, to imitate what can be observed in an empirical study (“Virtual Ecologist” approach; Zurell et al., 2010)?

\paragraph{Explanation}
% The ‘Design concepts’ element of the ODD protocol does not describe the model per se; i.e., it is not needed to replicate a model. However, these design concepts tend to be characteristic of ABMs, though certainly not exclusively. They may also be crucial to interpreting the output of a model, and they are not described well via traditional model description techniques such as equations and flow charts. Therefore, they are included in ODD as a kind of checklist to make sure that important model design decisions are made consciously and that readers are aware of these decisions (Railsback, 2001; Grimm and Railsback, 2005). For example, almost all ABMs include some kinds of adaptive traits, but if these traits do not use an explicit objective measure the ‘Objectives’ and perhaps ‘Prediction’ concepts are not relevant (though many ABMs include hidden or implicit predictions). Also, many ABMs do not include learning or collectives. Unused concepts can be omitted in the ODD description.

\paragraph{Other...?}
% There might be important concepts underlying the design of an ABM that are not included in the ODD protocol. If authors feel that it is important to understand a certain new concept to understand the design of their model, they should give it a short name, clearly announce it as a design concept not included in the ODD protocol, and present it at the end of the Design concepts element.

\subsection*{Initialization}
% What is the initial state of the model world, i.e., at time t = 0 of a simulation run? In detail, how many entities of what type are there initially, and what are the exact values of their state variables (or how were they set stochastically)? Is initialization always the same, or is it allowed to vary among simulations? Are the initial values chosen arbitrarily or based on data? References to those data should be provided.
% Model results cannot be accurately replicated unless the initial conditions are known. Different models, and different analyses using the same model, can of course depend quite differently on initial conditions. Sometimes the purpose of a model is to analyze consequences of its initial state, and other times modelers try hard to minimize the effect of initial conditions on results.
Native protocol file layout, but (hopefully!) able to read in other formats such as SBML \cite{Hucka2003} and CellML \cite{Cuellar2003}.

\subsection*{Input}
% Does the model use input from external sources such as data files or other models to represent processes that change over time?
Using input from external sources (data files or other models): there is potential to make a process manager that could handle this, but none are currently implemented. Coupling with a fluid-flow package would be great.

\subsection*{Submodels}
% What, in detail, are the submodels that represent the processes listed in ‘Process overview and scheduling’? What are the model parameters, their dimensions, and reference values? How were submodels designed or chosen, and how were they parameterized and then tested?
% See Grimm et al (2010) for further explanation
\paragraph{?}


% Results and Discussion can be combined.
\section*{Results}
% TODO

% In order to `sell' iDyno 2.0 this section would probably benefit from a set of mini examples showcasing/ demonstrating iDynoMiCS 2.0
% key selling points, we need to justify the claims we make in the introduction and show how easy and usefull iDynoMiCS 2.0 is. 

\subsection*{Case study: agent force functions}

\subsection*{Case study: filamentous organism and multi-grid-cell agents}


\section*{Discussion}
% TODO

\paragraph{Ease of comparison}
Since iDynoMiCS 2 already adopts the hybrid approach to modelling (discrete) cells in (continuous) concentration fields, it is straightforward from a design perspective to instead treat cells as a continuum. This "retrofit" allows easy comparison of the hybrid IbM approach with more traditional population-based approaches \cite{wanner1986}
% TODO Need more citations? Alpkvist & Klapper?
for the same system. Such calibration is already used often,
% TODO Need citation(s) here
but implementing this feature is, to our knowledge, the first time it has been offered \textit{within a single software package}.
% TODO Need to check this claim!

\paragraph{Modularity}
Aids customisability

\section*{Supporting Information}

% Include only the SI item label in the subsection heading. Use the \nameref{label} command to cite SI items in the text.


\section*{Acknowledgements}
% TODO Need to list funding sources.

\nolinenumbers

%\section*{References}
% Either type in your references using
% \begin{thebibliography}{}
% \bibitem{}
% Text
% \end{thebibliography}
%
% OR
%
% Compile your BiBTeX database using our plos2015.bst
% style file and paste the contents of your .bbl file
% here.

\bibliography{bibliography}

\end{document}

